% Options for packages loaded elsewhere
\PassOptionsToPackage{unicode}{hyperref}
\PassOptionsToPackage{hyphens}{url}
%
\documentclass[
]{article}
\usepackage{amsmath,amssymb}
\usepackage{lmodern}
\usepackage{iftex}
\ifPDFTeX
  \usepackage[T1]{fontenc}
  \usepackage[utf8]{inputenc}
  \usepackage{textcomp} % provide euro and other symbols
\else % if luatex or xetex
  \usepackage{unicode-math}
  \defaultfontfeatures{Scale=MatchLowercase}
  \defaultfontfeatures[\rmfamily]{Ligatures=TeX,Scale=1}
\fi
% Use upquote if available, for straight quotes in verbatim environments
\IfFileExists{upquote.sty}{\usepackage{upquote}}{}
\IfFileExists{microtype.sty}{% use microtype if available
  \usepackage[]{microtype}
  \UseMicrotypeSet[protrusion]{basicmath} % disable protrusion for tt fonts
}{}
\makeatletter
\@ifundefined{KOMAClassName}{% if non-KOMA class
  \IfFileExists{parskip.sty}{%
    \usepackage{parskip}
  }{% else
    \setlength{\parindent}{0pt}
    \setlength{\parskip}{6pt plus 2pt minus 1pt}}
}{% if KOMA class
  \KOMAoptions{parskip=half}}
\makeatother
\usepackage{xcolor}
\usepackage[margin=1in]{geometry}
\usepackage{color}
\usepackage{fancyvrb}
\newcommand{\VerbBar}{|}
\newcommand{\VERB}{\Verb[commandchars=\\\{\}]}
\DefineVerbatimEnvironment{Highlighting}{Verbatim}{commandchars=\\\{\}}
% Add ',fontsize=\small' for more characters per line
\usepackage{framed}
\definecolor{shadecolor}{RGB}{248,248,248}
\newenvironment{Shaded}{\begin{snugshade}}{\end{snugshade}}
\newcommand{\AlertTok}[1]{\textcolor[rgb]{0.94,0.16,0.16}{#1}}
\newcommand{\AnnotationTok}[1]{\textcolor[rgb]{0.56,0.35,0.01}{\textbf{\textit{#1}}}}
\newcommand{\AttributeTok}[1]{\textcolor[rgb]{0.77,0.63,0.00}{#1}}
\newcommand{\BaseNTok}[1]{\textcolor[rgb]{0.00,0.00,0.81}{#1}}
\newcommand{\BuiltInTok}[1]{#1}
\newcommand{\CharTok}[1]{\textcolor[rgb]{0.31,0.60,0.02}{#1}}
\newcommand{\CommentTok}[1]{\textcolor[rgb]{0.56,0.35,0.01}{\textit{#1}}}
\newcommand{\CommentVarTok}[1]{\textcolor[rgb]{0.56,0.35,0.01}{\textbf{\textit{#1}}}}
\newcommand{\ConstantTok}[1]{\textcolor[rgb]{0.00,0.00,0.00}{#1}}
\newcommand{\ControlFlowTok}[1]{\textcolor[rgb]{0.13,0.29,0.53}{\textbf{#1}}}
\newcommand{\DataTypeTok}[1]{\textcolor[rgb]{0.13,0.29,0.53}{#1}}
\newcommand{\DecValTok}[1]{\textcolor[rgb]{0.00,0.00,0.81}{#1}}
\newcommand{\DocumentationTok}[1]{\textcolor[rgb]{0.56,0.35,0.01}{\textbf{\textit{#1}}}}
\newcommand{\ErrorTok}[1]{\textcolor[rgb]{0.64,0.00,0.00}{\textbf{#1}}}
\newcommand{\ExtensionTok}[1]{#1}
\newcommand{\FloatTok}[1]{\textcolor[rgb]{0.00,0.00,0.81}{#1}}
\newcommand{\FunctionTok}[1]{\textcolor[rgb]{0.00,0.00,0.00}{#1}}
\newcommand{\ImportTok}[1]{#1}
\newcommand{\InformationTok}[1]{\textcolor[rgb]{0.56,0.35,0.01}{\textbf{\textit{#1}}}}
\newcommand{\KeywordTok}[1]{\textcolor[rgb]{0.13,0.29,0.53}{\textbf{#1}}}
\newcommand{\NormalTok}[1]{#1}
\newcommand{\OperatorTok}[1]{\textcolor[rgb]{0.81,0.36,0.00}{\textbf{#1}}}
\newcommand{\OtherTok}[1]{\textcolor[rgb]{0.56,0.35,0.01}{#1}}
\newcommand{\PreprocessorTok}[1]{\textcolor[rgb]{0.56,0.35,0.01}{\textit{#1}}}
\newcommand{\RegionMarkerTok}[1]{#1}
\newcommand{\SpecialCharTok}[1]{\textcolor[rgb]{0.00,0.00,0.00}{#1}}
\newcommand{\SpecialStringTok}[1]{\textcolor[rgb]{0.31,0.60,0.02}{#1}}
\newcommand{\StringTok}[1]{\textcolor[rgb]{0.31,0.60,0.02}{#1}}
\newcommand{\VariableTok}[1]{\textcolor[rgb]{0.00,0.00,0.00}{#1}}
\newcommand{\VerbatimStringTok}[1]{\textcolor[rgb]{0.31,0.60,0.02}{#1}}
\newcommand{\WarningTok}[1]{\textcolor[rgb]{0.56,0.35,0.01}{\textbf{\textit{#1}}}}
\usepackage{graphicx}
\makeatletter
\def\maxwidth{\ifdim\Gin@nat@width>\linewidth\linewidth\else\Gin@nat@width\fi}
\def\maxheight{\ifdim\Gin@nat@height>\textheight\textheight\else\Gin@nat@height\fi}
\makeatother
% Scale images if necessary, so that they will not overflow the page
% margins by default, and it is still possible to overwrite the defaults
% using explicit options in \includegraphics[width, height, ...]{}
\setkeys{Gin}{width=\maxwidth,height=\maxheight,keepaspectratio}
% Set default figure placement to htbp
\makeatletter
\def\fps@figure{htbp}
\makeatother
\setlength{\emergencystretch}{3em} % prevent overfull lines
\providecommand{\tightlist}{%
  \setlength{\itemsep}{0pt}\setlength{\parskip}{0pt}}
\setcounter{secnumdepth}{-\maxdimen} % remove section numbering
\newlength{\cslhangindent}
\setlength{\cslhangindent}{1.5em}
\newlength{\csllabelwidth}
\setlength{\csllabelwidth}{3em}
\newlength{\cslentryspacingunit} % times entry-spacing
\setlength{\cslentryspacingunit}{\parskip}
\newenvironment{CSLReferences}[2] % #1 hanging-ident, #2 entry spacing
 {% don't indent paragraphs
  \setlength{\parindent}{0pt}
  % turn on hanging indent if param 1 is 1
  \ifodd #1
  \let\oldpar\par
  \def\par{\hangindent=\cslhangindent\oldpar}
  \fi
  % set entry spacing
  \setlength{\parskip}{#2\cslentryspacingunit}
 }%
 {}
\usepackage{calc}
\newcommand{\CSLBlock}[1]{#1\hfill\break}
\newcommand{\CSLLeftMargin}[1]{\parbox[t]{\csllabelwidth}{#1}}
\newcommand{\CSLRightInline}[1]{\parbox[t]{\linewidth - \csllabelwidth}{#1}\break}
\newcommand{\CSLIndent}[1]{\hspace{\cslhangindent}#1}
\usepackage{booktabs}
\usepackage{longtable}
\usepackage{array}
\usepackage{multirow}
\usepackage{wrapfig}
\usepackage{float}
\usepackage{colortbl}
\usepackage{pdflscape}
\usepackage{tabu}
\usepackage{threeparttable}
\usepackage{threeparttablex}
\usepackage[normalem]{ulem}
\usepackage{makecell}
\usepackage{xcolor}
\ifLuaTeX
  \usepackage{selnolig}  % disable illegal ligatures
\fi
\IfFileExists{bookmark.sty}{\usepackage{bookmark}}{\usepackage{hyperref}}
\IfFileExists{xurl.sty}{\usepackage{xurl}}{} % add URL line breaks if available
\urlstyle{same} % disable monospaced font for URLs
\hypersetup{
  pdftitle={R Markdown},
  pdfauthor={programandoenr},
  hidelinks,
  pdfcreator={LaTeX via pandoc}}

\title{R Markdown}
\author{programandoenr}
\date{2022-09-09}

\begin{document}
\maketitle

\begin{figure}

\includegraphics[width=0.05\linewidth]{rmarkdown} \hfill{}

\caption{Logo}\label{fig:rmarkdown}
\end{figure}

\hypertarget{gruxe1fico-ggplot2}{%
\section{- Gráfico ggplot2}\label{gruxe1fico-ggplot2}}

Este gráfico es realizado con ggplot2:
\url{https://ggplot2.tidyverse.org/}

\hypertarget{datos}{%
\subsection{- Datos}\label{datos}}

\begin{Shaded}
\begin{Highlighting}[]
\FunctionTok{library}\NormalTok{(readxl)}
\NormalTok{note }\OtherTok{\textless{}{-}} \FunctionTok{read\_excel}\NormalTok{(}\StringTok{"note.xlsx"}\NormalTok{)}
\NormalTok{ma }\OtherTok{\textless{}{-}} \FunctionTok{source}\NormalTok{(}\StringTok{"ma.R"}\NormalTok{) }
\end{Highlighting}
\end{Shaded}

\hypertarget{gruxe1ficos}{%
\subsection{- Gráficos}\label{gruxe1ficos}}

\begin{Shaded}
\begin{Highlighting}[]
\FunctionTok{ggplot}\NormalTok{(not2) }\SpecialCharTok{+} 
     \FunctionTok{geom\_col}\NormalTok{(}\FunctionTok{aes}\NormalTok{(kk, ECTS, }\AttributeTok{fill =}\NormalTok{ kk, }\AttributeTok{width =}\NormalTok{ .}\DecValTok{9}\NormalTok{), }\AttributeTok{stat =} \StringTok{\textquotesingle{}identity\textquotesingle{}}\NormalTok{) }\SpecialCharTok{+} 
  \FunctionTok{scale\_fill\_viridis\_d}\NormalTok{(}\AttributeTok{labels =}\NormalTok{ lb, legend\_title }\OtherTok{\textless{}{-}} \StringTok{"Semester"}\NormalTok{) }\SpecialCharTok{+}
  \FunctionTok{scale\_x\_discrete}\NormalTok{(}\AttributeTok{labels =}\NormalTok{ lb) }\SpecialCharTok{+}
  \FunctionTok{guides}\NormalTok{(}\AttributeTok{col =} \FunctionTok{guide\_legend}\NormalTok{(}\AttributeTok{title =} \StringTok{"ECTS"}\NormalTok{)) }\SpecialCharTok{+}
  \FunctionTok{labs}\NormalTok{(}\AttributeTok{title =} \StringTok{\textquotesingle{}Credits Received | Créditos recibidos\textquotesingle{}}\NormalTok{,}
    \AttributeTok{x =} \StringTok{\textquotesingle{}Semester | Semestre\textquotesingle{}}\NormalTok{, }\AttributeTok{y =} \StringTok{"ECTS | Créditos recibidos"}\NormalTok{) }
\end{Highlighting}
\end{Shaded}

\includegraphics{project_files/figure-latex/unnamed-chunk-2-1.pdf}

\hypertarget{tabla}{%
\section{- Tabla}\label{tabla}}

Tabla de datos ordenada:

\begin{Shaded}
\begin{Highlighting}[]
\FunctionTok{library}\NormalTok{(DT)}
\FunctionTok{library}\NormalTok{(kableExtra)}
\NormalTok{tb }\OtherTok{\textless{}{-}}\NormalTok{ not2[}\DecValTok{1}\SpecialCharTok{:}\DecValTok{7}\NormalTok{, }\DecValTok{1}\SpecialCharTok{:}\DecValTok{4}\NormalTok{]}
\NormalTok{tb }\SpecialCharTok{\%\textgreater{}\%}
  \FunctionTok{kbl}\NormalTok{() }\SpecialCharTok{\%\textgreater{}\%}
  \FunctionTok{kable\_styling}\NormalTok{()}
\end{Highlighting}
\end{Shaded}

\begin{table}
\centering
\begin{tabular}[t]{l|l|r|r}
\hline
Semester & kk & ECTS & n\\
\hline
HS2018 & A & 22 & 4\\
\hline
FS2019 & B & 18 & 3\\
\hline
HS2019 & C & 18 & 5\\
\hline
FS2020 & D & 36 & 6\\
\hline
HS2020 & E & 33 & 6\\
\hline
FS2021 & F & 24 & 3\\
\hline
HS2021 & G & 15 & 3\\
\hline
\end{tabular}
\end{table}

Esta es una plantilla básica R-Markdown (Allaire et al. 2022)

\hypertarget{referencias}{%
\section*{- Referencias}\label{referencias}}
\addcontentsline{toc}{section}{- Referencias}

\hypertarget{refs}{}
\begin{CSLReferences}{1}{0}
\leavevmode\vadjust pre{\hypertarget{ref-R-rmarkdown}{}}%
Allaire, JJ, Yihui Xie, Jonathan McPherson, Javier Luraschi, Kevin
Ushey, Aron Atkins, Hadley Wickham, Joe Cheng, Winston Chang, and
Richard Iannone. 2022. \emph{Rmarkdown: Dynamic Documents for r}.
\url{https://CRAN.R-project.org/package=rmarkdown}.

\end{CSLReferences}

\end{document}
